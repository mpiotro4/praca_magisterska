%%%%%%%%%%%%%%%%%%%%%%%%%%%%%%%%%%%%%%%%%%%%%%%%%%%%%%%
%% Bachelor's & Master's Thesis Template             %%
%% Copyleft by Artur M. Brodzki & Piotr Woźniak      %%
%% Faculty of Electronics and Information Technology %%
%% Warsaw University of Technology, 2019-2020        %%
%%%%%%%%%%%%%%%%%%%%%%%%%%%%%%%%%%%%%%%%%%%%%%%%%%%%%%%

\documentclass[
    left=2.5cm,         % Sadly, generic margin parameter
    right=2.5cm,        % doesnt't work, as it is
    top=2.5cm,          % superseded by more specific
    bottom=3cm,         % left...bottom parameters.
    bindingoffset=6mm,  % Optional binding offset.
    nohyphenation=false % You may turn off hyphenation, if don't like.
]{eiti/eiti-thesis}

\langpol % Dla języka angielskiego mamy \langeng
\graphicspath{{img/}}             % Katalog z obrazkami.
\addbibresource{bibliografia.bib} % Plik .bib z bibliografią

\begin{document}

%--------------------------------------
% Strona tytułowa
%--------------------------------------
\MasterThesis % Dla pracy inżynierskiej mamy \EngineerThesis
\instytut{Instytut Automatyki i Informatyki Stosowanej}
\kierunek{Informatyka}
\specjalnosc{Inteligentne Systemy}
\title{
    Narzędzie do wspomagania automatyzacji testów oprogramowania
}
\engtitle{ % Tytuł po angielsku do angielskiego streszczenia
    A tool to support software testing automation
}
\author{Marcin Piotrowski}
\album{300346}
\promotor{dr inż. Andrzej Ratkowski}
\date{\the\year}
\maketitle

%--------------------------------------
% Streszczenie po polsku
%--------------------------------------
\cleardoublepage % Zaczynamy od nieparzystej strony
\streszczenie \lipsum[1-3]
\slowakluczowe testy, oprogramowanie, bpmn

%--------------------------------------
% Streszczenie po angielsku
%--------------------------------------
\newpage
\abstract \kant[1-3]
\keywords tests, software, bpm

%--------------------------------------
% Oświadczenie o autorstwie
%--------------------------------------
\cleardoublepage  % Zaczynamy od nieparzystej strony
\pagestyle{plain}
\makeauthorship

%--------------------------------------
% Spis treści
%--------------------------------------
\cleardoublepage % Zaczynamy od nieparzystej strony
\tableofcontents

%--------------------------------------
% Rozdziały
%--------------------------------------
\cleardoublepage % Zaczynamy od nieparzystej strony
\pagestyle{headings}

\newpage % Rozdziały zaczynamy od nowej strony.
\section{Wstęp}
We współczesnej inżynierii oprogramowania panuje przeświadczenie, że należy uznać, iż to co nie jest przetestowane nie działa. Świadczyć może o tym fakt, że testowanie stanowi ponad 50\% kosztów produkcji oprogramowania\cite{kumar2016impacts}. Wysoki koszt testowania spowodowany jest tym jak czasochłonny i złożony jest to proces\cite{jamil2016software}. Stanowi on wyzwanie dla wszelkiej maści specjalistów począwszy od analityków biznesowych skończywszy na testerach i programistach. Oczywiste zatem jest, że istnieje potrzeba aby proces ten choć w niewielkim stopniu usprawnić gdyż niesie to za sobą potencjalnie spore oszczędności.

Jednym z możliwych usprawnień jest zastępowanie testów manualnych testami automatycznymi. Takie podejście pomimo wielu zalet charakteryzuje się też tym że wymaga dużych nakładów pracy do wdrożenia gdyż wymaga ono stworzenia przez zespół deweloperski osobnego testującego programu\cite{kumar2016impacts}.

Głównym celem niniejszej pracy było opracowanie i zaimplementowanie aplikacji, która służy do automatycznego generowania kodu testów. Spośród wielu typów aplikacji zdecydowano że przedmiotem badań będą aplikacje sterowane procesami (PDA), których tematykę przybliżono w sekcji \ref{subsec:pda}. Nie odłącznym elementem PDA jest modelowanie i notacja procesów biznesowych (BPMN) i systemy zarządzania nimi (BPMS) dlatego odpowiednio w sekcjach \ref{subsec:bpmn} i \ref{subec:bpms} w ramach wstępu omówiono te zagadnienia. W sekcji \ref{subsec:cel} przedstawiono dokładny cel oraz zakres tej pracy.

\subsection{Aplikacje sterowane procesami (PDA)}
\label{subsec:pda}
Aplikacje sterowane procesami (ang. Process-Driven Applications - PDA) charakteryzują się tym, że znacząca część ich logiki biznesowej jest zaimplementowana  jako wykonywalny model procesu, zazwyczaj w ustandaryzowanej notacji BPMN. W odróżnieniu od tradycyjnych aplikacji gdzie logika jest zaimplementowana w kodzie, PDA umożliwiają modelowanie procesów w sposób graficzny, co ułatwia ich zrozumienie i modyfikację. Aplikacje sterowane procesami integrują różnorodne elementy, takie jak zarządzanie przepływem pracy, automatyzacja zadań oraz monitorowanie i optymalizacja procesów biznesowych. Dzięki temu organizacje mogą lepiej zarządzać swoimi operacjami, zwiększając efektywność oraz redukując błędy i koszty operacyjne\cite{Process-Driven-Applications-with-BPMN}.

\subsection{Modelowanie i Notacja Procesów 
\label{subsec:bpmn}
Biznesowych (BPMN)}
BPMN (ang. Business Process Model and Notation) jest standardem modelowania procesów biznesowych opracowanym przez Object Management Group (OMG) \cite{omg_bpmn}. Umożliwia on graficzne przedstawienie procesów biznesowych, co wspiera lepszą komunikację między analitykami biznesowymi a zespołami technicznymi. Dzięki standaryzowanej notacji, BPMN pozwala na precyzyjne definiowanie sekwencji działań, punktów decyzyjnych, przepływów danych oraz interakcji między różnymi uczestnikami procesów\cite{white_bpmn}. O tym jak bardzo uniwersalny jest ten standard może świadczyć fakt, że jest on szeroko stosowany w różnych branżach, od finansów\cite{finance_bpmn} po produkcję\cite{production_bpmn} i jest kluczowym narzędziem w zarządzaniu procesami biznesowymi\cite{dumas_bpmn}.

\subsection{Systemy zarządzania procesami biznesowymi 
\label{subec:bpms}
(BPMS)}
Implementacja PDA wiąże się z wykorzystaniem narzędzi do zarządzania procesami biznesowymi (ang. Business Process Management Systems - BPMS)\cite{techtarget_bpms}, które oferują funkcjonalności takie jak:
\begin{itemize}
\item Modelowanie procesów – tworzenie i edytowanie modeli procesów biznesowych.
\item Symulacja – testowanie modeli procesów przed ich wdrożeniem.
\item Wykonanie – uruchamianie i monitorowanie procesów w czasie rzeczywistym.
\item Analiza – zbieranie i analizowanie danych dotyczących wykonania procesów w celu ich optymalizacji.
\end{itemize}
Na rynku istnieje wiele różnych implementacji systemów do zarządzania procesami biznesowymi (ang. Business Process Management Systems - BPMS), które wspierają tworzenie i zarządzanie PDA. Przykładami takich systemów są Camunda\cite{camunda}, IBM Business Automation Workflow\cite{ibm_baw}, Appian\cite{appian}, Pega\cite{pega}, Oracle BPM Suite\cite{oracle_bpm}, Bizagi\cite{bizagi}, Bonita\cite{bonitasoft}, K2\cite{k2}, Microsoft Power Automate\cite{microsoft_power_automate} oraz Signavio\cite{signavio}. Każdy z tych systemów oferuje unikalne funkcje i możliwości, które mogą być dostosowane do specyficznych potrzeb organizacji.

\subsection{Cel i zakres pracy}
\label{subsec:cel}
Celem niniejszej pracy było zaimplementowanie narzędzia służącego do weryfikacji zgodności procesów biznesowych zapisanych w notacji BPMN z diagramami przypadków użycia w notacji UML. Zaimplementowana aplikacja miała na celu dostarczenie interesariuszom informacji czy proces biznesowy będący implementacją przypadku użycia spełnia stawiane przez niego założenia odnośnie elementów procesu i możliwych scenariuszy.


         % Wygodnie jest trzymać każdy rozdział w osobnym pliku.
\input{tex/2-Przegląd literatury}    % Umożliwia to również łatwą migrację do nowej wersji szablonu:
\newpage % Rozdziały zaczynamy od nowej strony.
\section{Code listings}

\lipsum[10]
% \addmargin pozwala na wcięcie kodu od lewej (tutaj: 6mm).
% Wcięcie pomaga ustawić kod tak, 
% aby numery linii nie były za bardzo na lewo. 
% Druga liczba oznacza wcięcie od prawej. 
\begin{addmargin}[6mm]{0mm}
\begin{lstlisting}[
    language=HTML,
    numbers=left,
    firstnumber=1,
    caption={\emph{Hello world} w HTML},
    aboveskip=0pt
]
<html>
  <head>
    <title>Hello world!</title>
  </head>
  <body>
    Hello world!
  </body>
</html>
\end{lstlisting}
\end{addmargin}

\lipsum[11]
% Dla dłuższych numerów linii
% potrzebne jest większe wcięcie.
\begin{addmargin}[10mm]{0mm}
\begin{lstlisting}[
    language=C++,
    numbers=left,
    firstnumber=147,
    caption={Generowanie sekwencji Collatza w języku C++},
    aboveskip=0pt
]
class Collatz {
  private:
    unsigned current_val_;
    void update_val() {
        if( current_val_ % 2 == 0 )
            current_val_ /= 2;
        else
            current_val_ = current_val_ * 3 + 1;
    }

  public:
    explicit Collatz(unsigned initial_value) : 
        current_val_(initial_value) {}
    void print_sequence() {
        unsigned i = 1;
        while( current_val_ > 1 ) {
            std::cout
                << "val " << i << " = " << current_val_
                << std::endl;
            update_val(); ++i;
        }
    }
};

int main() {
  // prints Collatz seqence, starting from 194375
  Collatz seq(194375);
  seq.print_sequence();
  return 0;
}
\end{lstlisting}
\end{addmargin}

\lipsum[12]
 % wystarczy podmienić swoje pliki main.tex i eiti-thesis.cls
                            % na nowe wersje, a cały tekst pracy pozostaje nienaruszony.

\newpage % Rozdziały zaczynamy od nowej strony
\section{Summatio}          % Można też pisać rozdziały w jednym pliku.
\lipsum[5-10]

%--------------------------------------------
% Literatura
%--------------------------------------------
\cleardoublepage % Zaczynamy od nieparzystej strony
\printbibliography

%--------------------------------------------
% Spisy (opcjonalne)
%--------------------------------------------
\newpage
\pagestyle{plain}

% Wykaz symboli i skrótów.
% Pamiętaj, żeby posortować symbole alfabetycznie
% we własnym zakresie. Ponieważ mało kto używa takiego wykazu,
% uznałem, że robienie automatycznie sortowanej listy
% na poziomie LaTeXa to za duży overkill.
% Makro \acronymlist generuje właściwy tytuł sekcji,
% w zależności od języka.
% Makro \acronym dodaje skrót/symbol do listy,
% zapewniając podstawowe formatowanie.
% //AB
\vspace{0.8cm}
\acronymlist
\acronym{EiTI}{Wydział Elektroniki i Technik Informacyjnych}
\acronym{PW}{Politechnika Warszawska}
\acronym{WEIRD}{ang. \emph{Western, Educated, Industrialized, Rich and Democratic}}

\listoffigurestoc     % Spis rysunków.
\vspace{1cm}          % vertical space
\listoftablestoc      % Spis tabel.
\vspace{1cm}          % vertical space
\listofappendicestoc  % Spis załączników

% Załączniki
\newpage
\appendix{Nazwa załącznika 1}
\lipsum[1-8]

\newpage
\appendix{Nazwa załącznika 2}
\lipsum[1-4]

% Używając powyższych spisów jako szablonu,
% możesz tu dodać swój własny wykaz bądź listę,
% np. spis algorytmów.

\end{document} % Dobranoc.
